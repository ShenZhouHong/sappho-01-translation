\section*{Preface to the Hong translation}
Out of all great poets, both ancient and modern --- the artistry and emotion of Sappho deserves especial remark. Writing from the Island of Lesbos around 600 B.C.E, her passionate poetry on her love of woman has given us the very word 'Lesbian' today. It is a tragedy on par with the destruction of Alexandria, that so few of her works have survived until the present age.

Sappho fragment 1 is one of these survivors. For the longest time, considered the only complete poem that is extant today, the \emph{Ode to Aprhodite} paints an vivid picture of unrequited love, filled with not just longing, but humor and wit. Beginning in the same manner as Homer's Iliad with an appeal to the Gods, Sappho not only laments her spurning lover, but also paints an surprisingly relatable picture of the Goddess Aphrodite, more teasing co-conspirator, than distant deity.

The translation itself is also of academic and linguistic interest, being particularly noted for the obscureness of the Aeolic dialect. In my translation of Sappho fragment 1, I have attempted to translate from first principles, rather than rely on the pre-prepared glosses offered to the rest of the class. In this process, I have encountered numerous uncommon words and obscure usages of poetic crasis, making the translation technically challenging.

As for the final form of my polished translation, I aimed for a more casual, free-verse rendition, where I emphasized fluency over linguistic accuracy, with particular care to reject archaic english in favor of a more conversational tone. For me, Sappho writes not to the linguist or academic, but to the heart of all that are lovesick or spurned, and it is my wish to make this poem accessible to all those who know the pain of a broken heart.
