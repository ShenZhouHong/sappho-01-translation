\section*{Translation notes and interlinear gloss}
This appendix of the translation is included in order to demonstrate the process of parsing and glossing the Aeolic Greek source text. The translation process was divided into 7 distinct stanzas. For each stanza, \emph{Translator's Notes} are added for any noteworthy interpretations, or scholarly commentary on particularly difficult clauses.

In addition to the \emph{Translator's Notes}, for the first stanza additional detail has been added in the form of detailed 'glosstables'. These tables contain exhaustive linguistic parsing of each word in the stanza, as well as their declensions/conjugations. In contrast with the prior Apollodorus translation, exhaustive 'glosstables' for each stanza have been omitted for the sake of brevity.

\subsection*{Stanza 1: Sappho's Prayer}
For this stanza, exhaustive 'glosstables' have been included in order to further elaborate on the parsing process. The glosstables templates are generated programmatically in Markdown using an Python.

\begin{quote}
  ποικιλόθρον’ ἀθανάτ Ἀφρόδιτα,\\
  παῖ Δίος δολόπλοκε, λίσσομαί σε,\\
  μή μ’ ἄσαισι μηδ’ ὀνίαισι δάμνα,\\
  πότνια, θῦμον,\\
\end{quote}

\begin{table}[H]
\begin{tabular}{@{}llll@{}}
\toprule
\textbf{Greek word} & \textbf{Type} & \textbf{Gloss}      & \textbf{Meaning}                 \\ \midrule
ποικιλόθρον'        & Adj.          & sg. fem. voc. & \textit{On richly-worked throne} \\
ἀθανάτ              & Adj.          & sg. fem. voc. & \textit{undying/immortal}        \\
Ἀφρόδιτα,           & Noun          & sg. voc.            & \textit{Aphrodite}               \\ \bottomrule
\end{tabular}
\end{table}

\begin{table}[H]
\begin{tabular}{@{}llll@{}}
\toprule
\textbf{Greek word} & \textbf{Type} & \textbf{Gloss}     & \textbf{Meaning}             \\ \midrule
παῖ                 & Noun          & sg. fem. voc.      & \textit{child}               \\
Δίος                & Noun          & sg. masc. gen.     & \textit{(of) Zeus}           \\
δολόπλοκε,          & Adj.          & sg. fem. voc       & \textit{weaving wiles}       \\
λίσσομαί            & Verb          & 1st. sg. pres. mp. & \textit{I pray (for myself)} \\
σε,                 & Pron.         & 2nd sg. acc.       & \textit{you}                 \\ \bottomrule
\end{tabular}
\end{table}

\begin{table}[H]
\begin{tabular}{@{}llll@{}}
\toprule
\textbf{Greek word} & \textbf{Type} & \textbf{Gloss} & \textbf{Meaning}                \\ \midrule
μή                  & Neg.          & N/A            & not                             \\
μ'                  & Pron.         & 1st sg. acc.   & myself                          \\
ἄσαισι              & Noun          & pl. fem. dat.  & to/with distress                \\
μηδ'                & Partic.       & N/A            & and not                         \\
ὀνίαισι             & Noun          & pl. fem. dat.  & to/with grief, sorrow, distress \\
δάμνα,              & Verb          & 3rd. sg.       & to overpower                    \\ \bottomrule
\end{tabular}
\end{table}

\begin{table}[H]
\begin{tabular}{@{}llll@{}}
\toprule
\textbf{Greek word} & \textbf{Type} & \textbf{Gloss}     & \textbf{Meaning} \\ \midrule
πότνια,             & Noun          & sg. fem. voc. & \textit{Queen}   \\
θῦμον,              & Noun          & sg. masc. acc      & \textit{soul}    \\ \bottomrule
\end{tabular}
\end{table}

\subsubsection*{Translator's Notes}
\begin{itemize}
  \item μ' stems from ἐγώ
  \item The Liddell and Scott also gives an variety of possible translations for ἄσαισι, stemming from ἄση. Alternative meanings can include: surfeit, loathing, nausea, distress, vexation, or even longing and desire.
  \item μηδ᾽ is taken as μηδέ, which is a particle of negation.
  \item δάμνα as stemming from δαμνάω instead of the etymologically similar μι-verb δάμνημι
\end{itemize}

\subsection*{Stanza 2: The Appeal to Arrive}
\begin{quote}
  ἀλλά τυίδ᾽ ἔλθ᾽, αἴποτα κἀτέρωτα\\
  τᾶσ ἔμασ αύδωσ αἴοισα πήλγι\\
  ἔκλυεσ πάτροσ δὲ δόμον λίποισα\\
  χρύσιον ἦλθεσ
\end{quote}

\subsubsection*{Translator's Notes}
\begin{itemize}
  \item τυίδ' stemming from τυῖδε
  \item ἔλθ' is probably a enclitic form of deponent verb ἔρχομαι
  \item Supposedly κἀτέρωτα is Aeolic for ἑτέρωθι.
  \item πήλοι taken as the Aeolic τηλοῦ, according to the Liddell and Scott.
  \item πάτρος is father in the genitive. This translation assumes the same usage as in the phrase τὰ πρὸς πατρός.
\end{itemize}

\subsection*{Stanza 3: Journey from Heaven to Earth}
\begin{quote}
  ἄρμ᾽ ὐποζεύξαια, κάλοι δέ σ᾽ ἆγον\\
  ὤκεεσ στροῦθοι περὶ γᾶσ μελαίνασ\\
  πύκνα δινεῦντεσ πτέῤ ἀπ᾽ ὠράνω\\
  αἴθεροσ διὰ μέσσω.
\end{quote}

\subsubsection*{Translator's Notes}
\begin{itemize}
  \item ὐπασδεύξαισα is compound, taking form of both ὑπο (lit. under) and the μι-verb ζεύγνυμι.
  \item δίννεντες was especially difficult to find in the dictionary, but eventually I took it as "to circle about"
  \item ἀπ' ὠράνωἴθερος appears to be some form of crasis, perhaps ὠράνου αἴθερος. It is glossed as "heaven" in my translation.
\end{itemize}

\subsection*{Stanza 4: Aphrodite's Arrival}
\begin{quote}
  αῖψα δ᾽ ἐχίκοντο, σὺ δ᾽, ὦ μάσαιρα\\
  μειδιάσαισ᾽ ἀθάνατῳ προσώπῳ,\\
  ἤρἐ ὄττι δηὖτε πέπονθα κὤττι\\
  δἦγτε κάλημι
\end{quote}

\subsubsection*{Translator's Notes}
\begin{itemize}
  \item Note that likewise μειδιαίσαισ is not standard Attic, but rather Aeolic of μειδιᾶν participle in aorist feminist singular.
  \item "Smiling, immortal face'd" is indeed an exceptionally ugly translation. However, this is just the rough translation, done with emphasis of speed and preservation of word-order. The final polished translation will be much nicer.
  \item κὤττι is the crasis form of καὶ ὄττι
  \item ἤρε Aeolic for ἐρωτᾶν?
  \item δηὖτε is poetic crasis for δὴ αὖτε
  \item κάλημμι is Attic of καλέω
\end{itemize}

\subsection*{Stanza 5: Query for the Heartbreaker}
\begin{quote}
  κὤττι μοι μάλιστα θέλω γένεσθαι\\
  μαινόλᾳ θύμῳ, τίνα δηὖτε πείθω\\
  μαῖσ ἄγην ἐσ σὰν φιλότατα τίσ τ, ὦ\\
  Πσάπφ᾽, ἀδίκηει;
\end{quote}

\subsubsection*{Translator's Notes}
\begin{itemize}
  \item And once again, κὤττι is the crasis form of καὶ ὄττι
  \item μάλιστα is given as the superlative of μάλα
  \item δηὖτε is poetic crasis for δὴ αὖτε
  \item Taking ἄγην as the present infinitive form of ἀγάω
\end{itemize}

\subsection*{Stanza 6: Aphrodite's Consolation}
\begin{quote}
  καὶ γάρ αἰ φεύγει, ταχέωσ διώξει,\\
  αἰ δὲ δῶρα μὴ δέκετ ἀλλά δώσει,\\
  αἰ δὲ μὴ φίλει ταχέωσ φιλήσει,\\
  κωὐκ ἐθέλοισα.
\end{quote}

\subsubsection*{Translator's Notes}
\begin{itemize}
  \item Here the sigma which denotes the future tense is merged in διώξει.
  \item Thankfully, the only complicated part of this stanza was the usage of κωὐκ, the crasis form of καῖ οὐκ
  \item The Aeolic ἐθέλοισα is the Attic ἐθελοῦσα. Note that it is feminine, hence the unrequited lover is a woman.
\end{itemize}

\subsection*{Stanza 7: Sappho's wish}
\begin{quote}
  ἔλθε μοι καὶ νῦν, χαλεπᾶν δὲ λῦσον\\
  ἐκ μερίμναν ὄσσα δέ μοι τέλεσσαι\\
  θῦμοσ ἰμμέρρει τέλεσον, σὐ δ᾽ αὔτα\\
  σύμμαχοσ ἔσσο.
\end{quote}

\subsubsection*{Translator's Notes}
\begin{itemize}
  \item The scholar C.W. Conrad notes that the ἐκ is in tmesis, where it actually belongs to ἐκλῦσον, but is cut off in some reason. However translating it as a simple preposition seems to work out fine.
  \item Taking τέλεσσαι as the aorist optative.
  \item ἰμέρρει is another Aeolic superlative, most likely the Attic ἱμείρει.
\end{itemize}
